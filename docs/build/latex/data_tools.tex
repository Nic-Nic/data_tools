%% Generated by Sphinx.
\def\sphinxdocclass{report}
\documentclass[letterpaper,10pt,english]{sphinxmanual}
\ifdefined\pdfpxdimen
   \let\sphinxpxdimen\pdfpxdimen\else\newdimen\sphinxpxdimen
\fi \sphinxpxdimen=.75bp\relax

\usepackage[utf8]{inputenc}
\ifdefined\DeclareUnicodeCharacter
 \ifdefined\DeclareUnicodeCharacterAsOptional
  \DeclareUnicodeCharacter{"00A0}{\nobreakspace}
  \DeclareUnicodeCharacter{"2500}{\sphinxunichar{2500}}
  \DeclareUnicodeCharacter{"2502}{\sphinxunichar{2502}}
  \DeclareUnicodeCharacter{"2514}{\sphinxunichar{2514}}
  \DeclareUnicodeCharacter{"251C}{\sphinxunichar{251C}}
  \DeclareUnicodeCharacter{"2572}{\textbackslash}
 \else
  \DeclareUnicodeCharacter{00A0}{\nobreakspace}
  \DeclareUnicodeCharacter{2500}{\sphinxunichar{2500}}
  \DeclareUnicodeCharacter{2502}{\sphinxunichar{2502}}
  \DeclareUnicodeCharacter{2514}{\sphinxunichar{2514}}
  \DeclareUnicodeCharacter{251C}{\sphinxunichar{251C}}
  \DeclareUnicodeCharacter{2572}{\textbackslash}
 \fi
\fi
\usepackage{cmap}
\usepackage[T1]{fontenc}
\usepackage{amsmath,amssymb,amstext}
\usepackage{babel}
\usepackage{times}
\usepackage[Bjarne]{fncychap}
\usepackage[dontkeepoldnames]{sphinx}

\usepackage{geometry}

% Include hyperref last.
\usepackage{hyperref}
% Fix anchor placement for figures with captions.
\usepackage{hypcap}% it must be loaded after hyperref.
% Set up styles of URL: it should be placed after hyperref.
\urlstyle{same}

\addto\captionsenglish{\renewcommand{\figurename}{Fig.}}
\addto\captionsenglish{\renewcommand{\tablename}{Table}}
\addto\captionsenglish{\renewcommand{\literalblockname}{Listing}}

\addto\captionsenglish{\renewcommand{\literalblockcontinuedname}{continued from previous page}}
\addto\captionsenglish{\renewcommand{\literalblockcontinuesname}{continues on next page}}

\addto\extrasenglish{\def\pageautorefname{page}}

\setcounter{tocdepth}{4}
\setcounter{secnumdepth}{4}


\title{data\_tools Documentation}
\date{May 24, 2018}
\release{0.0.2}
\author{Nicolas Palacio}
\newcommand{\sphinxlogo}{\vbox{}}
\renewcommand{\releasename}{Release}
\makeindex

\begin{document}

\maketitle
\sphinxtableofcontents
\phantomsection\label{\detokenize{index::doc}}


Collection of Python functions and classes designed to make a
Computational Biologist’s life easier.

Copyright (C) 2018 Nicolàs Palacio

Contact: \sphinxhref{mailto:nicolaspalacio91@gmail.com}{nicolaspalacio91@gmail.com}

GNU-GLPv3:
This program is free software: you can redistribute it and/or modify it
under the terms of the GNU General Public License as published by the
Free Software Foundation.

This program is distributed in the hope that it will be useful, but
WITHOUT ANY WARRANTY; without even the implied warranty of
MERCHANTABILITY or FITNESS FOR A PARTICULAR PURPOSE. See the GNU General
Public License for more details.

A full copy of the GNU General Public License can be found on file
\sphinxhref{../../../LICENSE.md}{LICENSE.md}. If not, see
\sphinxurl{http://www.gnu.org/licenses/}.


\chapter{Dependencies}
\label{\detokenize{index:dependencies}}\label{\detokenize{index:data-tools-s-reference}}\begin{itemize}
\item {} 
\sphinxhref{http://www.numpy.org/}{NumPy}

\item {} 
\sphinxhref{https://matplotlib.org/}{Matplotlib}

\item {} 
\sphinxhref{https://pandas.pydata.org/}{Pandas}

\end{itemize}


\chapter{Installation}
\label{\detokenize{index:installation}}
First download/clone \sphinxcode{data\_tools} from the
\sphinxhref{https://github.com/Nic-Nic/data\_tools.git}{GitHub repository}.
From the terminal:

\fvset{hllines={, ,}}%
\begin{sphinxVerbatim}[commandchars=\\\{\}]
git clone https://github.com/Nic\PYGZhy{}Nic/data\PYGZus{}tools.git
\PYG{n+nb}{cd} data\PYGZus{}tools
\end{sphinxVerbatim}

Then you can install it by running \sphinxcode{setup.py} as follows:

\fvset{hllines={, ,}}%
\begin{sphinxVerbatim}[commandchars=\\\{\}]
python setup.py sdist
\end{sphinxVerbatim}

Or using \sphinxcode{pip}:

\fvset{hllines={, ,}}%
\begin{sphinxVerbatim}[commandchars=\\\{\}]
pip install .
\end{sphinxVerbatim}


\chapter{Reference}
\label{\detokenize{index:reference}}\phantomsection\label{\detokenize{plots:module-data_tools.plots}}\index{data\_tools.plots (module)}

\section{data\_tools.plots}
\label{\detokenize{plots::doc}}\label{\detokenize{plots:data-tools-plots}}
Plotting functions module.
\index{volcano() (in module data\_tools.plots)}

\begin{fulllineitems}
\phantomsection\label{\detokenize{plots:data_tools.plots.volcano}}\pysiglinewithargsret{\sphinxcode{data\_tools.plots.}\sphinxbfcode{volcano}}{\emph{logfc}, \emph{logpval}, \emph{thr\_pval=0.05}, \emph{thr\_fc=2.0}, \emph{c=('C0'}, \emph{'C1')}, \emph{legend=True}, \emph{title=None}, \emph{filename=None}, \emph{figsize=None}}{}
Generates a volcano plot from the differential expression data
provided.
\begin{itemize}
\item {} \begin{description}
\item[{Arguments:}] \leavevmode\begin{itemize}
\item {} 
\sphinxstyleemphasis{logfc} {[}list{]}: Or any iterable type. Contains the log
(usually base 2) fold-change values. Must have the same length
as \sphinxstyleemphasis{logpval}.

\item {} 
\sphinxstyleemphasis{logpval} {[}list{]}: Or any iterable type. Contains the -log
p-values (usually base 10). Must have the same length as
\sphinxstyleemphasis{logfc}.

\item {} 
\sphinxstyleemphasis{thr\_pval} {[}float{]}: Optional, \sphinxcode{0.05} by default. Specifies
the p-value (non log-transformed) threshold to consider a
measurement as significantly differentially expressed.

\item {} 
\sphinxstyleemphasis{thr\_fc} {[}float{]}: Optional, \sphinxcode{2}. by default. Specifies the
FC (non log-transformed) threshold to consider a measurement
as significantly differentially expressed.

\item {} 
\sphinxstyleemphasis{c} {[}tuple{]}: Optional, \sphinxcode{('C0', 'C1')} by default (matplotlib
default colors). Any iterable containing two color arguments
tolerated by matplotlib (e.g.: \sphinxcode{{[}'r', 'b'{]}} for red and
blue). First one is used for non-significant points, second
for the significant ones.

\item {} 
\sphinxstyleemphasis{legend} {[}bool{]}: Optional, \sphinxcode{True} by default. Indicates
whether to show the plot legend or not.

\item {} 
\sphinxstyleemphasis{title} {[}str{]}: Optional, \sphinxcode{None} by default. Defines the plot
title.

\item {} 
\sphinxstyleemphasis{filename} {[}str{]}: Optional, \sphinxcode{None} by default. If passed,
indicates the file name or path where to store the figure.
Format must be specified (e.g.: .png, .pdf, etc)

\item {} 
\sphinxstyleemphasis{figsize} {[}tuple{]}: Optional, \sphinxcode{None} by default (default
matplotlib size). Any iterable containing two values denoting
the figure size (in inches) as {[}width, height{]}.

\end{itemize}

\end{description}

\item {} \begin{description}
\item[{Returns:}] \leavevmode\begin{itemize}
\item {} 
{[}matplotlib.figure.Figure{]}: Figure object containing the
volcano plot.

\end{itemize}

\end{description}

\item {} \begin{description}
\item[{Examples:}] \leavevmode
\fvset{hllines={, ,}}%
\begin{sphinxVerbatim}[commandchars=\\\{\}]
\PYG{g+gp}{\PYGZgt{}\PYGZgt{}\PYGZgt{} }\PYG{n}{volcano}\PYG{p}{(}\PYG{n}{my\PYGZus{}log\PYGZus{}fc}\PYG{p}{,} \PYG{n}{my\PYGZus{}log\PYGZus{}pval}\PYG{p}{)}
\end{sphinxVerbatim}

\noindent\sphinxincludegraphics{{volcano_example}.png}

\end{description}

\end{itemize}

\end{fulllineitems}

\index{piano\_consensus() (in module data\_tools.plots)}

\begin{fulllineitems}
\phantomsection\label{\detokenize{plots:data_tools.plots.piano_consensus}}\pysiglinewithargsret{\sphinxcode{data\_tools.plots.}\sphinxbfcode{piano\_consensus}}{\emph{df}, \emph{nchar=40}, \emph{boxes=True}, \emph{title=None}, \emph{filename=None}, \emph{figsize=None}}{}
Generates a GSEA consensus score plot like R package \sphinxcode{piano}’s
\sphinxcode{consensusScores} function, but prettier.
The main input is assumed to be a \sphinxcode{pandas.DataFrame} whose data
is the same as the \sphinxcode{rankMat} from the result of
\sphinxcode{consensusScores}.
\begin{itemize}
\item {} \begin{description}
\item[{Arguments:}] \leavevmode\begin{itemize}
\item {} 
\sphinxstyleemphasis{df} {[}pandas.DataFrame{]}: Values contained correspond to the
scores of the gene-sets (consensus and each individual
methods). Index must contain the gene-set labels. Columns are
assumed to be \sphinxcode{ConsRank} (ignored), \sphinxcode{ConsScore} followed
by the individual methods (e.g.: \sphinxcode{mean}, \sphinxcode{median},
\sphinxcode{sum}, etc).

\item {} 
\sphinxstyleemphasis{nchar} {[}int{]}: Optional, \sphinxcode{40} by default. Number of string
characters of the gene-set labels of the plot.

\item {} 
\sphinxstyleemphasis{boxes} {[}bool{]}: Optional, \sphinxcode{True} by default. Determines
whether to show the boxplots of the gene-sets or not.

\item {} 
\sphinxstyleemphasis{title} {[}str{]}: Optional, \sphinxcode{None} by default. Defines the plot
title.

\item {} 
\sphinxstyleemphasis{filename} {[}str{]}: Optional, \sphinxcode{None} by default. If passed,
indicates the file name or path where to store the figure.
Format must be specified (e.g.: .png, .pdf, etc)

\item {} 
\sphinxstyleemphasis{figsize} {[}tuple{]}: Optional, \sphinxcode{None} by default (default
matplotlib size). Any iterable containing two values denoting
the figure size (in inches) as {[}width, height{]}.

\end{itemize}

\end{description}

\item {} \begin{description}
\item[{Returns:}] \leavevmode\begin{itemize}
\item {} 
{[}\sphinxstyleemphasis{matplotlib.figure.Figure}{]}: the figure object containing a
combination of box and scatter plots of the gene-set scores.

\end{itemize}

\end{description}

\item {} \begin{description}
\item[{Examples:}] \leavevmode
\fvset{hllines={, ,}}%
\begin{sphinxVerbatim}[commandchars=\\\{\}]
\PYG{g+gp}{\PYGZgt{}\PYGZgt{}\PYGZgt{} }\PYG{n}{piano\PYGZus{}consensus}\PYG{p}{(}\PYG{n}{df}\PYG{p}{,} \PYG{n}{figsize}\PYG{o}{=}\PYG{p}{[}\PYG{l+m+mi}{7}\PYG{p}{,} \PYG{l+m+mi}{8}\PYG{p}{]}\PYG{p}{)}
\end{sphinxVerbatim}

\noindent\sphinxincludegraphics{{piano_consensus_example}.png}

\end{description}

\end{itemize}

\end{fulllineitems}

\phantomsection\label{\detokenize{sets:module-data_tools.sets}}\index{data\_tools.sets (module)}

\section{data\_tools.sets}
\label{\detokenize{sets:data-tools-sets}}\label{\detokenize{sets::doc}}
Set operations module.
\index{in\_all() (in module data\_tools.sets)}

\begin{fulllineitems}
\phantomsection\label{\detokenize{sets:data_tools.sets.in_all}}\pysiglinewithargsret{\sphinxcode{data\_tools.sets.}\sphinxbfcode{in\_all}}{\emph{x}, \emph{N}}{}
Checks if a vector \sphinxstyleemphasis{x} is present in all sets contained in a list
\sphinxstyleemphasis{N}.
\begin{itemize}
\item {} \begin{description}
\item[{Arguments:}] \leavevmode\begin{itemize}
\item {} 
\sphinxstyleemphasis{x} {[}tuple{]}: Or any hashable type as long as is the same
contained in the sets of \sphinxstyleemphasis{N}.

\item {} 
\sphinxstyleemphasis{N} {[}list{]}: Or any iterable type containing {[}set{]} objects.

\end{itemize}

\end{description}

\item {} \begin{description}
\item[{Returns:}] \leavevmode\begin{itemize}
\item {} 
{[}bool{]}: \sphinxcode{True} if \sphinxstyleemphasis{x} is found in all sets of \sphinxstyleemphasis{N}, \sphinxcode{False}
otherwise.

\end{itemize}

\end{description}

\item {} \begin{description}
\item[{Examples:}] \leavevmode
\fvset{hllines={, ,}}%
\begin{sphinxVerbatim}[commandchars=\\\{\}]
\PYG{g+gp}{\PYGZgt{}\PYGZgt{}\PYGZgt{} }\PYG{n}{N} \PYG{o}{=} \PYG{p}{[}\PYG{p}{\PYGZob{}}\PYG{p}{(}\PYG{l+m+mi}{0}\PYG{p}{,} \PYG{l+m+mi}{0}\PYG{p}{)}\PYG{p}{,} \PYG{p}{(}\PYG{l+m+mi}{0}\PYG{p}{,} \PYG{l+m+mi}{1}\PYG{p}{)}\PYG{p}{\PYGZcb{}}\PYG{p}{,} \PYG{c+c1}{\PYGZsh{} \PYGZlt{}\PYGZhy{} set A}
\PYG{g+gp}{... }     \PYG{p}{\PYGZob{}}\PYG{p}{(}\PYG{l+m+mi}{0}\PYG{p}{,} \PYG{l+m+mi}{0}\PYG{p}{)}\PYG{p}{,} \PYG{p}{(}\PYG{l+m+mi}{1}\PYG{p}{,} \PYG{l+m+mi}{1}\PYG{p}{)}\PYG{p}{,} \PYG{p}{(}\PYG{l+m+mi}{1}\PYG{p}{,} \PYG{l+m+mi}{0}\PYG{p}{)}\PYG{p}{\PYGZcb{}}\PYG{p}{]} \PYG{c+c1}{\PYGZsh{} \PYGZlt{}\PYGZhy{} set B}
\PYG{g+gp}{\PYGZgt{}\PYGZgt{}\PYGZgt{} }\PYG{n}{x} \PYG{o}{=} \PYG{p}{(}\PYG{l+m+mi}{0}\PYG{p}{,} \PYG{l+m+mi}{0}\PYG{p}{)}
\PYG{g+gp}{\PYGZgt{}\PYGZgt{}\PYGZgt{} }\PYG{n}{in\PYGZus{}all}\PYG{p}{(}\PYG{n}{x}\PYG{p}{,} \PYG{n}{N}\PYG{p}{)}
\PYG{g+go}{True}
\PYG{g+gp}{\PYGZgt{}\PYGZgt{}\PYGZgt{} }\PYG{n}{y} \PYG{o}{=} \PYG{p}{(}\PYG{l+m+mi}{0}\PYG{p}{,} \PYG{l+m+mi}{1}\PYG{p}{)}
\PYG{g+gp}{\PYGZgt{}\PYGZgt{}\PYGZgt{} }\PYG{n}{in\PYGZus{}all}\PYG{p}{(}\PYG{n}{y}\PYG{p}{,} \PYG{n}{N}\PYG{p}{)}
\PYG{g+go}{False}
\end{sphinxVerbatim}

\end{description}

\end{itemize}

\end{fulllineitems}

\index{bit\_or() (in module data\_tools.sets)}

\begin{fulllineitems}
\phantomsection\label{\detokenize{sets:data_tools.sets.bit_or}}\pysiglinewithargsret{\sphinxcode{data\_tools.sets.}\sphinxbfcode{bit\_or}}{\emph{a}, \emph{b}}{}
Returns the bit operation OR between two bit-strings \sphinxstyleemphasis{a} and \sphinxstyleemphasis{b}.
NOTE: \sphinxstyleemphasis{a} and \sphinxstyleemphasis{b} must have the same size.
\begin{itemize}
\item {} \begin{description}
\item[{Arguments:}] \leavevmode\begin{itemize}
\item {} 
\sphinxstyleemphasis{a} {[}tuple{]}: Or any iterable type.

\item {} 
\sphinxstyleemphasis{b} {[}tuple{]}: Or any iterable type.

\end{itemize}

\end{description}

\item {} \begin{description}
\item[{Returns:}] \leavevmode\begin{itemize}
\item {} 
{[}tuple{]}: OR operation between \sphinxstyleemphasis{a} and \sphinxstyleemphasis{b} element-wise.

\end{itemize}

\end{description}

\item {} \begin{description}
\item[{Examples:}] \leavevmode
\fvset{hllines={, ,}}%
\begin{sphinxVerbatim}[commandchars=\\\{\}]
\PYG{g+gp}{\PYGZgt{}\PYGZgt{}\PYGZgt{} }\PYG{n}{a}\PYG{p}{,} \PYG{n}{b} \PYG{o}{=} \PYG{p}{(}\PYG{l+m+mi}{0}\PYG{p}{,} \PYG{l+m+mi}{0}\PYG{p}{,} \PYG{l+m+mi}{1}\PYG{p}{)}\PYG{p}{,} \PYG{p}{(}\PYG{l+m+mi}{1}\PYG{p}{,} \PYG{l+m+mi}{0}\PYG{p}{,} \PYG{l+m+mi}{1}\PYG{p}{)}
\PYG{g+gp}{\PYGZgt{}\PYGZgt{}\PYGZgt{} }\PYG{n}{bit\PYGZus{}or}\PYG{p}{(}\PYG{n}{a}\PYG{p}{,} \PYG{n}{b}\PYG{p}{)}
\PYG{g+go}{(1, 0, 1)}
\end{sphinxVerbatim}

\end{description}

\end{itemize}

\end{fulllineitems}

\index{multi\_union() (in module data\_tools.sets)}

\begin{fulllineitems}
\phantomsection\label{\detokenize{sets:data_tools.sets.multi_union}}\pysiglinewithargsret{\sphinxcode{data\_tools.sets.}\sphinxbfcode{multi\_union}}{\emph{N}}{}
Returns the union set of all sets contained in a list \sphinxstyleemphasis{N}.
\begin{itemize}
\item {} \begin{description}
\item[{Arguments:}] \leavevmode\begin{itemize}
\item {} 
\sphinxstyleemphasis{N} {[}list{]}: Or any iterable type containing {[}set{]} objects.

\end{itemize}

\end{description}

\item {} \begin{description}
\item[{Returns:}] \leavevmode\begin{itemize}
\item {} 
{[}set{]}: The union of all sets contained in \sphinxstyleemphasis{N}.

\end{itemize}

\end{description}

\item {} \begin{description}
\item[{Examples:}] \leavevmode
\fvset{hllines={, ,}}%
\begin{sphinxVerbatim}[commandchars=\\\{\}]
\PYG{g+gp}{\PYGZgt{}\PYGZgt{}\PYGZgt{} }\PYG{n}{A} \PYG{o}{=} \PYG{p}{\PYGZob{}}\PYG{l+m+mi}{1}\PYG{p}{,} \PYG{l+m+mi}{3}\PYG{p}{,} \PYG{l+m+mi}{5}\PYG{p}{\PYGZcb{}}
\PYG{g+gp}{\PYGZgt{}\PYGZgt{}\PYGZgt{} }\PYG{n}{B} \PYG{o}{=} \PYG{p}{\PYGZob{}}\PYG{l+m+mi}{0}\PYG{p}{,} \PYG{l+m+mi}{1}\PYG{p}{,} \PYG{l+m+mi}{2}\PYG{p}{\PYGZcb{}}
\PYG{g+gp}{\PYGZgt{}\PYGZgt{}\PYGZgt{} }\PYG{n}{C} \PYG{o}{=} \PYG{p}{\PYGZob{}}\PYG{l+m+mi}{0}\PYG{p}{,} \PYG{l+m+mi}{2}\PYG{p}{,} \PYG{l+m+mi}{5}\PYG{p}{\PYGZcb{}}
\PYG{g+gp}{\PYGZgt{}\PYGZgt{}\PYGZgt{} }\PYG{n}{multi\PYGZus{}union}\PYG{p}{(}\PYG{p}{[}\PYG{n}{A}\PYG{p}{,} \PYG{n}{B}\PYG{p}{,} \PYG{n}{C}\PYG{p}{]}\PYG{p}{)}
\PYG{g+go}{set([0, 1, 2, 3, 5])}
\end{sphinxVerbatim}

\end{description}

\end{itemize}

\end{fulllineitems}

\index{find\_min() (in module data\_tools.sets)}

\begin{fulllineitems}
\phantomsection\label{\detokenize{sets:data_tools.sets.find_min}}\pysiglinewithargsret{\sphinxcode{data\_tools.sets.}\sphinxbfcode{find\_min}}{\emph{A}}{}
Finds and returns the subset of vectors whose sum is minimum from a
given set \sphinxstyleemphasis{A}.
\begin{itemize}
\item {} \begin{description}
\item[{Arguments:}] \leavevmode\begin{itemize}
\item {} 
\sphinxstyleemphasis{A} {[}set{]}: Set of vectors ({[}tuple{]} or any iterable).

\end{itemize}

\end{description}

\item {} \begin{description}
\item[{Returns:}] \leavevmode\begin{itemize}
\item {} 
{[}set{]}: Subset of vectors in \sphinxstyleemphasis{A} whose sum is minimum.

\end{itemize}

\end{description}

\item {} \begin{description}
\item[{Examples:}] \leavevmode
\fvset{hllines={, ,}}%
\begin{sphinxVerbatim}[commandchars=\\\{\}]
\PYG{g+gp}{\PYGZgt{}\PYGZgt{}\PYGZgt{} }\PYG{n}{A} \PYG{o}{=} \PYG{p}{\PYGZob{}}\PYG{p}{(}\PYG{l+m+mi}{0}\PYG{p}{,} \PYG{l+m+mi}{1}\PYG{p}{,} \PYG{l+m+mi}{1}\PYG{p}{)}\PYG{p}{,} \PYG{p}{(}\PYG{l+m+mi}{0}\PYG{p}{,} \PYG{l+m+mi}{1}\PYG{p}{,} \PYG{l+m+mi}{0}\PYG{p}{)}\PYG{p}{,} \PYG{p}{(}\PYG{l+m+mi}{1}\PYG{p}{,} \PYG{l+m+mi}{0}\PYG{p}{,} \PYG{l+m+mi}{0}\PYG{p}{)}\PYG{p}{,} \PYG{p}{(}\PYG{l+m+mi}{1}\PYG{p}{,} \PYG{l+m+mi}{1}\PYG{p}{,} \PYG{l+m+mi}{1}\PYG{p}{)}\PYG{p}{\PYGZcb{}}
\PYG{g+gp}{\PYGZgt{}\PYGZgt{}\PYGZgt{} }\PYG{n}{find\PYGZus{}min}\PYG{p}{(}\PYG{n}{A}\PYG{p}{)}
\PYG{g+go}{set([(0, 1, 0), (1, 0, 0)])}
\end{sphinxVerbatim}

\end{description}

\end{itemize}

\end{fulllineitems}

\phantomsection\label{\detokenize{strings:module-data_tools.strings}}\index{data\_tools.strings (module)}

\section{data\_tools.strings}
\label{\detokenize{strings::doc}}\label{\detokenize{strings:data-tools-strings}}
String operations module.
\index{is\_numeric() (in module data\_tools.strings)}

\begin{fulllineitems}
\phantomsection\label{\detokenize{strings:data_tools.strings.is_numeric}}\pysiglinewithargsret{\sphinxcode{data\_tools.strings.}\sphinxbfcode{is\_numeric}}{\emph{s}}{}
Determines if a string can be considered a numeric value. NaN is
also considered, since it is float type.
\begin{itemize}
\item {} \begin{description}
\item[{Arguments:}] \leavevmode\begin{itemize}
\item {} 
\sphinxstyleemphasis{s} {[}str{]}: String to be evaluated.

\end{itemize}

\end{description}

\item {} \begin{description}
\item[{Returns:}] \leavevmode\begin{itemize}
\item {} 
{[}bool{]}: \sphinxcode{True}/\sphinxcode{False} depending if the condition is
satisfied.

\end{itemize}

\end{description}

\item {} \begin{description}
\item[{Examples:}] \leavevmode
\fvset{hllines={, ,}}%
\begin{sphinxVerbatim}[commandchars=\\\{\}]
\PYG{g+gp}{\PYGZgt{}\PYGZgt{}\PYGZgt{} }\PYG{n}{is\PYGZus{}numeric}\PYG{p}{(}\PYG{l+s+s1}{\PYGZsq{}}\PYG{l+s+s1}{4}\PYG{l+s+s1}{\PYGZsq{}}\PYG{p}{)}
\PYG{g+go}{True}
\PYG{g+gp}{\PYGZgt{}\PYGZgt{}\PYGZgt{} }\PYG{n}{is\PYGZus{}numeric}\PYG{p}{(}\PYG{l+s+s1}{\PYGZsq{}}\PYG{l+s+s1}{\PYGZhy{}3.2}\PYG{l+s+s1}{\PYGZsq{}}\PYG{p}{)}
\PYG{g+go}{True}
\PYG{g+gp}{\PYGZgt{}\PYGZgt{}\PYGZgt{} }\PYG{n}{is\PYGZus{}numeric}\PYG{p}{(}\PYG{l+s+s1}{\PYGZsq{}}\PYG{l+s+s1}{number}\PYG{l+s+s1}{\PYGZsq{}}\PYG{p}{)}
\PYG{g+go}{False}
\PYG{g+gp}{\PYGZgt{}\PYGZgt{}\PYGZgt{} }\PYG{n}{is\PYGZus{}numeric}\PYG{p}{(}\PYG{l+s+s1}{\PYGZsq{}}\PYG{l+s+s1}{NaN}\PYG{l+s+s1}{\PYGZsq{}}\PYG{p}{)}
\PYG{g+go}{True}
\end{sphinxVerbatim}

\end{description}

\end{itemize}

\end{fulllineitems}

\index{join\_str\_lists() (in module data\_tools.strings)}

\begin{fulllineitems}
\phantomsection\label{\detokenize{strings:data_tools.strings.join_str_lists}}\pysiglinewithargsret{\sphinxcode{data\_tools.strings.}\sphinxbfcode{join\_str\_lists}}{\emph{a}, \emph{b}, \emph{sep=''}}{}
Joins element-wise two lists (or any 1D iterable) of strings with a
given separator (if provided). Length of the input lists must be
equal.
\begin{itemize}
\item {} \begin{description}
\item[{Arguments:}] \leavevmode\begin{itemize}
\item {} 
\sphinxstyleemphasis{a} {[}list{]}: Contains the first elements {[}str{]} of the joint
strings.

\item {} 
\sphinxstyleemphasis{b} {[}list{]}: Contains the second elements {[}str{]} of the joint
strings.

\item {} 
\sphinxstyleemphasis{sep} {[}str{]}: Optional \sphinxcode{'{'}} (non separated) by default.
Determines the separator between the joint strings.

\end{itemize}

\end{description}

\item {} \begin{description}
\item[{Returns:}] \leavevmode\begin{itemize}
\item {} 
{[}list{]}: List of the joint strings.

\end{itemize}

\end{description}

\item {} \begin{description}
\item[{Example:}] \leavevmode
\fvset{hllines={, ,}}%
\begin{sphinxVerbatim}[commandchars=\\\{\}]
\PYG{g+gp}{\PYGZgt{}\PYGZgt{}\PYGZgt{} }\PYG{n}{a} \PYG{o}{=} \PYG{p}{[}\PYG{l+s+s1}{\PYGZsq{}}\PYG{l+s+s1}{a}\PYG{l+s+s1}{\PYGZsq{}}\PYG{p}{,} \PYG{l+s+s1}{\PYGZsq{}}\PYG{l+s+s1}{b}\PYG{l+s+s1}{\PYGZsq{}}\PYG{p}{]}
\PYG{g+gp}{\PYGZgt{}\PYGZgt{}\PYGZgt{} }\PYG{n}{b} \PYG{o}{=} \PYG{p}{[}\PYG{l+s+s1}{\PYGZsq{}}\PYG{l+s+s1}{1}\PYG{l+s+s1}{\PYGZsq{}}\PYG{p}{,} \PYG{l+s+s1}{\PYGZsq{}}\PYG{l+s+s1}{2}\PYG{l+s+s1}{\PYGZsq{}}\PYG{p}{]}
\PYG{g+gp}{\PYGZgt{}\PYGZgt{}\PYGZgt{} }\PYG{n}{join\PYGZus{}str\PYGZus{}lists}\PYG{p}{(}\PYG{n}{a}\PYG{p}{,} \PYG{n}{b}\PYG{p}{,} \PYG{n}{sep}\PYG{o}{=}\PYG{l+s+s1}{\PYGZsq{}}\PYG{l+s+s1}{\PYGZus{}}\PYG{l+s+s1}{\PYGZsq{}}\PYG{p}{)}
\PYG{g+go}{[\PYGZsq{}a\PYGZus{}1\PYGZsq{}, \PYGZsq{}b\PYGZus{}2\PYGZsq{}]}
\end{sphinxVerbatim}

\end{description}

\end{itemize}

\end{fulllineitems}



\renewcommand{\indexname}{Python Module Index}
\begin{sphinxtheindex}
\def\bigletter#1{{\Large\sffamily#1}\nopagebreak\vspace{1mm}}
\bigletter{d}
\item {\sphinxstyleindexentry{data\_tools.plots}}\sphinxstyleindexpageref{plots:\detokenize{module-data_tools.plots}}
\item {\sphinxstyleindexentry{data\_tools.sets}}\sphinxstyleindexpageref{sets:\detokenize{module-data_tools.sets}}
\item {\sphinxstyleindexentry{data\_tools.strings}}\sphinxstyleindexpageref{strings:\detokenize{module-data_tools.strings}}
\end{sphinxtheindex}

\renewcommand{\indexname}{Index}
\printindex
\end{document}