%% Generated by Sphinx.
\def\sphinxdocclass{report}
\documentclass[letterpaper,10pt,english]{sphinxmanual}
\ifdefined\pdfpxdimen
   \let\sphinxpxdimen\pdfpxdimen\else\newdimen\sphinxpxdimen
\fi \sphinxpxdimen=.75bp\relax

\usepackage[utf8]{inputenc}
\ifdefined\DeclareUnicodeCharacter
 \ifdefined\DeclareUnicodeCharacterAsOptional
  \DeclareUnicodeCharacter{"00A0}{\nobreakspace}
  \DeclareUnicodeCharacter{"2500}{\sphinxunichar{2500}}
  \DeclareUnicodeCharacter{"2502}{\sphinxunichar{2502}}
  \DeclareUnicodeCharacter{"2514}{\sphinxunichar{2514}}
  \DeclareUnicodeCharacter{"251C}{\sphinxunichar{251C}}
  \DeclareUnicodeCharacter{"2572}{\textbackslash}
 \else
  \DeclareUnicodeCharacter{00A0}{\nobreakspace}
  \DeclareUnicodeCharacter{2500}{\sphinxunichar{2500}}
  \DeclareUnicodeCharacter{2502}{\sphinxunichar{2502}}
  \DeclareUnicodeCharacter{2514}{\sphinxunichar{2514}}
  \DeclareUnicodeCharacter{251C}{\sphinxunichar{251C}}
  \DeclareUnicodeCharacter{2572}{\textbackslash}
 \fi
\fi
\usepackage{cmap}
\usepackage[T1]{fontenc}
\usepackage{amsmath,amssymb,amstext}
\usepackage{babel}
\usepackage{times}
\usepackage[Bjarne]{fncychap}
\usepackage[dontkeepoldnames]{sphinx}

\usepackage{geometry}

% Include hyperref last.
\usepackage{hyperref}
% Fix anchor placement for figures with captions.
\usepackage{hypcap}% it must be loaded after hyperref.
% Set up styles of URL: it should be placed after hyperref.
\urlstyle{same}

\addto\captionsenglish{\renewcommand{\figurename}{Fig.}}
\addto\captionsenglish{\renewcommand{\tablename}{Table}}
\addto\captionsenglish{\renewcommand{\literalblockname}{Listing}}

\addto\captionsenglish{\renewcommand{\literalblockcontinuedname}{continued from previous page}}
\addto\captionsenglish{\renewcommand{\literalblockcontinuesname}{continues on next page}}

\addto\extrasenglish{\def\pageautorefname{page}}

\setcounter{tocdepth}{4}
\setcounter{secnumdepth}{4}


\title{data\_tools Documentation}
\date{Jul 06, 2018}
\release{0.0.4}
\author{Nicolàs Palacio}
\newcommand{\sphinxlogo}{\vbox{}}
\renewcommand{\releasename}{Release}
\makeindex

\begin{document}

\maketitle
\sphinxtableofcontents
\phantomsection\label{\detokenize{index::doc}}


Data tools: a collection of Python functions and classes designed to
make data scientists’ life easier.

Copyright (C) 2018 Nicolàs Palacio

Contact: \sphinxhref{mailto:nicolaspalacio91@gmail.com}{nicolaspalacio91@gmail.com}

GNU-GLPv3:
This program is free software: you can redistribute it and/or modify it
under the terms of the GNU General Public License as published by the
Free Software Foundation.

This program is distributed in the hope that it will be useful, but
WITHOUT ANY WARRANTY; without even the implied warranty of
MERCHANTABILITY or FITNESS FOR A PARTICULAR PURPOSE. See the GNU General
Public License for more details.

A full copy of the GNU General Public License can be found on file
\sphinxhref{../../../LICENSE.md}{LICENSE.md}. If not, see
\sphinxurl{http://www.gnu.org/licenses/}.


\chapter{Installation}
\label{\detokenize{index:installation}}\label{\detokenize{index:data-tools-reference}}
First download/clone \sphinxcode{data\_tools} from the
\sphinxhref{https://github.com/Nic-Nic/data\_tools.git}{GitHub repository}.
From the terminal:

\fvset{hllines={, ,}}%
\begin{sphinxVerbatim}[commandchars=\\\{\}]
git clone https://github.com/Nic\PYGZhy{}Nic/data\PYGZus{}tools.git
\PYG{n+nb}{cd} data\PYGZus{}tools
\end{sphinxVerbatim}

Then you can install it by running \sphinxcode{setup.py} as follows:

\fvset{hllines={, ,}}%
\begin{sphinxVerbatim}[commandchars=\\\{\}]
python setup.py sdist
\end{sphinxVerbatim}

Or using \sphinxcode{pip}:

\fvset{hllines={, ,}}%
\begin{sphinxVerbatim}[commandchars=\\\{\}]
pip install .
\end{sphinxVerbatim}


\chapter{Module reference}
\label{\detokenize{index:module-reference}}\phantomsection\label{\detokenize{diffusion:module-data_tools.diffusion}}\index{data\_tools.diffusion (module)}

\section{data\_tools.diffusion}
\label{\detokenize{diffusion:data-tools-diffusion}}\label{\detokenize{diffusion::doc}}
Diffusion solvers module.


\subsection{Reference}
\label{\detokenize{diffusion:reference}}\index{euler\_explicit1D() (in module data\_tools.diffusion)}

\begin{fulllineitems}
\phantomsection\label{\detokenize{diffusion:data_tools.diffusion.euler_explicit1D}}\pysiglinewithargsret{\sphinxcode{data\_tools.diffusion.}\sphinxbfcode{euler\_explicit1D}}{\emph{x0}, \emph{dt}, \emph{dx2}, \emph{d=1}, \emph{bcs='periodic'}}{}
Computes diffusion on a 1D space over a time-step using Euler
explicit method
\begin{itemize}
\item {} \begin{description}
\item[{Arguments:}] \leavevmode\begin{itemize}
\item {} 
\sphinxstyleemphasis{x0} {[}{]}: .

\item {} 
\sphinxstyleemphasis{dt} {[}{]}: .

\item {} 
\sphinxstyleemphasis{dx2} {[}{]}: .

\item {} 
\sphinxstyleemphasis{d} {[}{]}: .

\item {} 
\sphinxstyleemphasis{bcs} {[}{]}: .

\end{itemize}

\end{description}

\item {} \begin{description}
\item[{Returns:}] \leavevmode\begin{itemize}
\item {} 
{[}{]}: .

\end{itemize}

\end{description}

\end{itemize}

\end{fulllineitems}

\phantomsection\label{\detokenize{models:module-data_tools.models}}\index{data\_tools.models (module)}

\section{data\_tools.models}
\label{\detokenize{models:data-tools-models}}\label{\detokenize{models::doc}}
Model classes module.


\subsection{Dependencies}
\label{\detokenize{models:dependencies}}\begin{itemize}
\item {} 
\sphinxhref{http://www.numpy.org/}{NumPy}

\item {} 
\sphinxhref{https://matplotlib.org/}{Matplotlib}

\item {} 
\sphinxhref{https://pandas.pydata.org/}{Pandas}

\item {} 
\sphinxhref{http://scikit-learn.org/stable/index.html}{Scikit-learn}

\end{itemize}


\subsection{Reference}
\label{\detokenize{models:reference}}\index{Lasso (class in data\_tools.models)}

\begin{fulllineitems}
\phantomsection\label{\detokenize{models:data_tools.models.Lasso}}\pysiglinewithargsret{\sphinxbfcode{class }\sphinxcode{data\_tools.models.}\sphinxbfcode{Lasso}}{\emph{Cs=500}, \emph{cv=10}, \emph{sampler='skf'}, \emph{solver='liblinear'}, \emph{**kwargs}}{}
Wrapper class inheriting from
\sphinxcode{sklearn.linear\_model.LogisticRegressionCV} with L1
regularization.
\begin{itemize}
\item {} \begin{description}
\item[{Arguments:}] \leavevmode\begin{itemize}
\item {} 
\sphinxstyleemphasis{Cs} {[}int{]}: Optional, \sphinxcode{500} by default. Integer or list of
float values of regularization parameters to test. If an
integer is passed, it will determine the number of values
taken from a logarithmic scale between \sphinxcode{1e-4} and \sphinxcode{1e4}.
Note that the value of the parameter is defined as the inverse
of the regularization strength.

\item {} 
\sphinxstyleemphasis{cv} {[}int{]}: Optional, \sphinxcode{10} by default. Denotes the number of
cross validation (CV) folds.

\item {} 
\sphinxstyleemphasis{sampler} {[}str{]}: Optional, \sphinxcode{'skf'} by default. Determines
which sampling method is used to generate the test and
training sets for CV. Methods available are K-Fold (\sphinxcode{'kf'}),
Shuffle Split (\sphinxcode{'ss'}) and their stratified variants
(\sphinxcode{'skf'} and \sphinxcode{'sss'} respectively).

\item {} 
\sphinxstyleemphasis{solver} {[}str{]}: Optional, \sphinxcode{'liblinear'} by default.
Determines which solver algorithm to use. Note that L1
regularization can only be handled by \sphinxcode{'liblinear'} and
\sphinxcode{'saga'}. Additionally if the classification is multinomial,
only the latter option is available.

\item {} 
\sphinxstyleemphasis{**kwargs}: Optional. Any other keyword argument accepted by
the \sphinxcode{sklearn.linear\_model.LogisticRegressionCV} class.

\end{itemize}

Other keyword arguments and functions available from the parent
class \sphinxcode{LogisticRegressionCV} can be fount in \sphinxhref{http://scikit-learn.org/stable/modules/generated/sklearn.linear\_model.LogisticRegressionCV.html}{Scikit-Learn’s
reference}.

\end{description}

\end{itemize}
\index{fit\_data() (data\_tools.models.Lasso method)}

\begin{fulllineitems}
\phantomsection\label{\detokenize{models:data_tools.models.Lasso.fit_data}}\pysiglinewithargsret{\sphinxbfcode{fit\_data}}{\emph{x}, \emph{y}, \emph{silent=False}}{}
Fits the data to the logistic model.
\begin{itemize}
\item {} \begin{description}
\item[{Arguments:}] \leavevmode\begin{itemize}
\item {} 
\sphinxstyleemphasis{x} {[}pandas.DataFrame{]}: Contains the values/measurements
{[}float{]} of the features (columns) for each
sample/replicate (rows).

\item {} 
\sphinxstyleemphasis{y} {[}pandas.Series{]}: List or any iterable containing the
observed class of each sample (must have the same order as
in \sphinxstyleemphasis{x}).

\item {} 
\sphinxstyleemphasis{silent} {[}bool{]}: Optional, \sphinxcode{False} by default.
Determines whether messages are printed or not.

\end{itemize}

\end{description}

\end{itemize}

\end{fulllineitems}

\index{plot\_coef() (data\_tools.models.Lasso method)}

\begin{fulllineitems}
\phantomsection\label{\detokenize{models:data_tools.models.Lasso.plot_coef}}\pysiglinewithargsret{\sphinxbfcode{plot\_coef}}{\emph{filename=None}, \emph{figsize=None}}{}
Plots the non-zero coefficients for the fitted predictor
features.
\begin{itemize}
\item {} \begin{description}
\item[{Arguments:}] \leavevmode\begin{itemize}
\item {} 
\sphinxstyleemphasis{filename} {[}str{]}: Optional, \sphinxcode{None} by default. If
passed, indicates the file name or path where to store the
figure. Format must be specified (e.g.: .png, .pdf, etc)

\item {} 
\sphinxstyleemphasis{figsize} {[}tuple{]}: Optional, \sphinxcode{None} by default (default
matplotlib size). Any iterable containing two values
denoting the figure size (in inches) as {[}width, height{]}.

\end{itemize}

\end{description}

\item {} \begin{description}
\item[{Returns:}] \leavevmode\begin{itemize}
\item {} 
{[}matplotlib.figure.Figure{]}: Figure object containing the
bar plot of the non-zero coefficients.

\end{itemize}

\end{description}

\end{itemize}

\end{fulllineitems}

\index{plot\_score() (data\_tools.models.Lasso method)}

\begin{fulllineitems}
\phantomsection\label{\detokenize{models:data_tools.models.Lasso.plot_score}}\pysiglinewithargsret{\sphinxbfcode{plot\_score}}{\emph{filename=None}, \emph{figsize=None}}{}
Plots the mean score across all folds obtained during CV.
The optimum C parameter chosen and its score are highlighted.
\begin{itemize}
\item {} \begin{description}
\item[{Arguments:}] \leavevmode\begin{itemize}
\item {} 
\sphinxstyleemphasis{filename} {[}str{]}: Optional, \sphinxcode{None} by default. If
passed, indicates the file name or path where to store the
figure. Format must be specified (e.g.: .png, .pdf, etc)

\item {} 
\sphinxstyleemphasis{figsize} {[}tuple{]}: Optional, \sphinxcode{None} by default (default
matplotlib size). Any iterable containing two values
denoting the figure size (in inches) as {[}width, height{]}.

\end{itemize}

\end{description}

\item {} \begin{description}
\item[{Returns:}] \leavevmode\begin{itemize}
\item {} 
{[}matplotlib.figure.Figure{]}: Figure object containing the
score plot.

\end{itemize}

\end{description}

\end{itemize}

\end{fulllineitems}


\end{fulllineitems}

\phantomsection\label{\detokenize{plots:module-data_tools.plots}}\index{data\_tools.plots (module)}

\section{data\_tools.plots}
\label{\detokenize{plots::doc}}\label{\detokenize{plots:data-tools-plots}}
Plotting functions module.


\subsection{Dependencies}
\label{\detokenize{plots:dependencies}}\begin{itemize}
\item {} 
\sphinxhref{http://www.numpy.org/}{NumPy}

\item {} 
\sphinxhref{https://matplotlib.org/}{Matplotlib}

\item {} 
\sphinxhref{https://pandas.pydata.org/}{Pandas}

\item {} 
\sphinxhref{https://www.scipy.org/}{SciPy}

\end{itemize}


\subsection{Reference}
\label{\detokenize{plots:reference}}\index{density() (in module data\_tools.plots)}

\begin{fulllineitems}
\phantomsection\label{\detokenize{plots:data_tools.plots.density}}\pysiglinewithargsret{\sphinxcode{data\_tools.plots.}\sphinxbfcode{density}}{\emph{df}, \emph{cvf=0.25}, \emph{title=None}, \emph{filename=None}, \emph{figsize=None}}{}
Generates a density plot of the values on a data frame (row-wise).
\begin{itemize}
\item {} \begin{description}
\item[{Arguments:}] \leavevmode\begin{itemize}
\item {} 
\sphinxstyleemphasis{df} {[}pandas.DataFrame{]}: Contains the values to generate the
plot. Each row is considered as an individual sample while
each column contains a measured value.

\item {} 
\sphinxstyleemphasis{cvf} {[}float{]}: Optional, \sphinxcode{0.25} by default. Co-variance
factor used in the gaussian kernel estimation. A higher value
increases the smoothness.

\item {} 
\sphinxstyleemphasis{title} {[}str{]}: Optional, \sphinxcode{None} by default. Defines the plot
title.

\item {} 
\sphinxstyleemphasis{filename} {[}str{]}: Optional, \sphinxcode{None} by default. If passed,
indicates the file name or path where to store the figure.
Format must be specified (e.g.: .png, .pdf, etc)

\item {} 
\sphinxstyleemphasis{figsize} {[}tuple{]}: Optional, \sphinxcode{None} by default (default
matplotlib size). Any iterable containing two values denoting
the figure size (in inches) as {[}width, height{]}.

\end{itemize}

\end{description}

\item {} \begin{description}
\item[{Returns:}] \leavevmode\begin{itemize}
\item {} 
{[}\sphinxstyleemphasis{matplotlib.figure.Figure}{]}: the figure object containing the
density plot.

\end{itemize}

\end{description}

\end{itemize}

\end{fulllineitems}

\index{piano\_consensus() (in module data\_tools.plots)}

\begin{fulllineitems}
\phantomsection\label{\detokenize{plots:data_tools.plots.piano_consensus}}\pysiglinewithargsret{\sphinxcode{data\_tools.plots.}\sphinxbfcode{piano\_consensus}}{\emph{df}, \emph{nchar=40}, \emph{boxes=True}, \emph{title=None}, \emph{filename=None}, \emph{figsize=None}}{}
Generates a GSEA consensus score plot like R package \sphinxcode{piano}’s
\sphinxcode{consensusScores} function, but prettier.
The main input is assumed to be a \sphinxcode{pandas.DataFrame} whose data
is the same as the \sphinxcode{rankMat} from the result of
\sphinxcode{consensusScores}.
\begin{itemize}
\item {} \begin{description}
\item[{Arguments:}] \leavevmode\begin{itemize}
\item {} 
\sphinxstyleemphasis{df} {[}pandas.DataFrame{]}: Values contained correspond to the
scores of the gene-sets (consensus and each individual
methods). Index must contain the gene-set labels. Columns are
assumed to be \sphinxcode{ConsRank} (ignored), \sphinxcode{ConsScore} followed
by the individual methods (e.g.: \sphinxcode{mean}, \sphinxcode{median},
\sphinxcode{sum}, etc).

\item {} 
\sphinxstyleemphasis{nchar} {[}int{]}: Optional, \sphinxcode{40} by default. Number of string
characters of the gene-set labels of the plot.

\item {} 
\sphinxstyleemphasis{boxes} {[}bool{]}: Optional, \sphinxcode{True} by default. Determines
whether to show the boxplots of the gene-sets or not.

\item {} 
\sphinxstyleemphasis{title} {[}str{]}: Optional, \sphinxcode{None} by default. Defines the plot
title.

\item {} 
\sphinxstyleemphasis{filename} {[}str{]}: Optional, \sphinxcode{None} by default. If passed,
indicates the file name or path where to store the figure.
Format must be specified (e.g.: .png, .pdf, etc)

\item {} 
\sphinxstyleemphasis{figsize} {[}tuple{]}: Optional, \sphinxcode{None} by default (default
matplotlib size). Any iterable containing two values denoting
the figure size (in inches) as {[}width, height{]}.

\end{itemize}

\end{description}

\item {} \begin{description}
\item[{Returns:}] \leavevmode\begin{itemize}
\item {} 
{[}\sphinxstyleemphasis{matplotlib.figure.Figure}{]}: the figure object containing a
combination of box and scatter plots of the gene-set scores.

\end{itemize}

\end{description}

\item {} \begin{description}
\item[{Example:}] \leavevmode
\fvset{hllines={, ,}}%
\begin{sphinxVerbatim}[commandchars=\\\{\}]
\PYG{g+gp}{\PYGZgt{}\PYGZgt{}\PYGZgt{} }\PYG{n}{piano\PYGZus{}consensus}\PYG{p}{(}\PYG{n}{df}\PYG{p}{,} \PYG{n}{figsize}\PYG{o}{=}\PYG{p}{[}\PYG{l+m+mi}{7}\PYG{p}{,} \PYG{l+m+mi}{8}\PYG{p}{]}\PYG{p}{)}
\end{sphinxVerbatim}

\noindent{\hspace*{\fill}\sphinxincludegraphics[scale=0.6]{{piano_consensus_example}.png}\hspace*{\fill}}

\end{description}

\end{itemize}

\end{fulllineitems}

\index{venn() (in module data\_tools.plots)}

\begin{fulllineitems}
\phantomsection\label{\detokenize{plots:data_tools.plots.venn}}\pysiglinewithargsret{\sphinxcode{data\_tools.plots.}\sphinxbfcode{venn}}{\emph{N, labels={[}'A', 'B', 'C', 'D'{]}, c={[}'C0', 'C1', 'C2', 'C3'{]}, title=None, filename=None, figsize=None}}{}
Plots a Venn diagram from a list of sets \sphinxstyleemphasis{N}. Number of sets must be
between 2 and 4 (inclusive).
\begin{itemize}
\item {} \begin{description}
\item[{Arguments:}] \leavevmode\begin{itemize}
\item {} 
\sphinxstyleemphasis{N} {[}list{]}: Or any iterable type containing {[}set{]} objects.

\item {} 
\sphinxstyleemphasis{labels} {[}list{]}: Optional, \sphinxcode{{[}'A', 'B', 'C', 'D'{]}} by
default. Labels for the sets following the same order as
provided in \sphinxstyleemphasis{N}.

\item {} 
\sphinxstyleemphasis{c} {[}list{]}: Optional, \sphinxcode{{[}'C0', 'C1' 'C2', 'C3'{]}} by default
(matplotlib default colors). Any iterable containing color
arguments tolerated by matplotlib (e.g.: \sphinxcode{{[}'r', 'b'{]}} for
red and blue). Must contain at least the same number of
elements as \sphinxstyleemphasis{N} (if more are provided, they will be ignored).

\item {} 
\sphinxstyleemphasis{title} {[}str{]}: Optional, \sphinxcode{None} by default. Defines the plot
title.

\item {} 
\sphinxstyleemphasis{filename} {[}str{]}: Optional, \sphinxcode{None} by default. If passed,
indicates the file name or path where to store the figure.
Format must be specified (e.g.: .png, .pdf, etc)

\item {} 
\sphinxstyleemphasis{figsize} {[}tuple{]}: Optional, \sphinxcode{None} by default (default
matplotlib size). Any iterable containing two values denoting
the figure size (in inches) as {[}width, height{]}.

\end{itemize}

\end{description}

\item {} \begin{description}
\item[{Returns:}] \leavevmode\begin{itemize}
\item {} 
{[}\sphinxstyleemphasis{matplotlib.figure.Figure}{]}: the figure object containing a
combination of box and scatter plots of the gene-set scores.

\end{itemize}

\end{description}

\item {} \begin{description}
\item[{Example:}] \leavevmode
\fvset{hllines={, ,}}%
\begin{sphinxVerbatim}[commandchars=\\\{\}]
\PYG{g+gp}{\PYGZgt{}\PYGZgt{}\PYGZgt{} }\PYG{n}{N} \PYG{o}{=} \PYG{p}{[}\PYG{p}{\PYGZob{}}\PYG{l+m+mi}{0}\PYG{p}{,} \PYG{l+m+mi}{1}\PYG{p}{\PYGZcb{}}\PYG{p}{,} \PYG{p}{\PYGZob{}}\PYG{l+m+mi}{2}\PYG{p}{,} \PYG{l+m+mi}{3}\PYG{p}{\PYGZcb{}}\PYG{p}{,} \PYG{p}{\PYGZob{}}\PYG{l+m+mi}{1}\PYG{p}{,} \PYG{l+m+mi}{3}\PYG{p}{,} \PYG{l+m+mi}{4}\PYG{p}{\PYGZcb{}}\PYG{p}{]} \PYG{c+c1}{\PYGZsh{} Sets A, B, C}
\PYG{g+gp}{\PYGZgt{}\PYGZgt{}\PYGZgt{} }\PYG{n}{venn}\PYG{p}{(}\PYG{n}{N}\PYG{p}{)}
\end{sphinxVerbatim}

\noindent{\hspace*{\fill}\sphinxincludegraphics[scale=1.0]{{venn_example}.png}\hspace*{\fill}}

\end{description}

\end{itemize}

\end{fulllineitems}

\index{volcano() (in module data\_tools.plots)}

\begin{fulllineitems}
\phantomsection\label{\detokenize{plots:data_tools.plots.volcano}}\pysiglinewithargsret{\sphinxcode{data\_tools.plots.}\sphinxbfcode{volcano}}{\emph{logfc}, \emph{logpval}, \emph{thr\_pval=0.05}, \emph{thr\_fc=2.0}, \emph{c=('C0'}, \emph{'C1')}, \emph{legend=True}, \emph{title=None}, \emph{filename=None}, \emph{figsize=None}}{}
Generates a volcano plot from the differential expression data
provided.
\begin{itemize}
\item {} \begin{description}
\item[{Arguments:}] \leavevmode\begin{itemize}
\item {} 
\sphinxstyleemphasis{logfc} {[}list{]}: Or any iterable type. Contains the log
(usually base 2) fold-change values. Must have the same length
as \sphinxstyleemphasis{logpval}.

\item {} 
\sphinxstyleemphasis{logpval} {[}list{]}: Or any iterable type. Contains the -log
p-values (usually base 10). Must have the same length as
\sphinxstyleemphasis{logfc}.

\item {} 
\sphinxstyleemphasis{thr\_pval} {[}float{]}: Optional, \sphinxcode{0.05} by default. Specifies
the p-value (non log-transformed) threshold to consider a
measurement as significantly differentially expressed.

\item {} 
\sphinxstyleemphasis{thr\_fc} {[}float{]}: Optional, \sphinxcode{2}. by default. Specifies the
FC (non log-transformed) threshold to consider a measurement
as significantly differentially expressed.

\item {} 
\sphinxstyleemphasis{c} {[}tuple{]}: Optional, \sphinxcode{('C0', 'C1')} by default (matplotlib
default colors). Any iterable containing two color arguments
tolerated by matplotlib (e.g.: \sphinxcode{{[}'r', 'b'{]}} for red and
blue). First one is used for non-significant points, second
for the significant ones.

\item {} 
\sphinxstyleemphasis{legend} {[}bool{]}: Optional, \sphinxcode{True} by default. Indicates
whether to show the plot legend or not.

\item {} 
\sphinxstyleemphasis{title} {[}str{]}: Optional, \sphinxcode{None} by default. Defines the plot
title.

\item {} 
\sphinxstyleemphasis{filename} {[}str{]}: Optional, \sphinxcode{None} by default. If passed,
indicates the file name or path where to store the figure.
Format must be specified (e.g.: .png, .pdf, etc)

\item {} 
\sphinxstyleemphasis{figsize} {[}tuple{]}: Optional, \sphinxcode{None} by default (default
matplotlib size). Any iterable containing two values denoting
the figure size (in inches) as {[}width, height{]}.

\end{itemize}

\end{description}

\item {} \begin{description}
\item[{Returns:}] \leavevmode\begin{itemize}
\item {} 
{[}matplotlib.figure.Figure{]}: Figure object containing the
volcano plot.

\end{itemize}

\end{description}

\item {} \begin{description}
\item[{Example:}] \leavevmode
\fvset{hllines={, ,}}%
\begin{sphinxVerbatim}[commandchars=\\\{\}]
\PYG{g+gp}{\PYGZgt{}\PYGZgt{}\PYGZgt{} }\PYG{n}{volcano}\PYG{p}{(}\PYG{n}{my\PYGZus{}log\PYGZus{}fc}\PYG{p}{,} \PYG{n}{my\PYGZus{}log\PYGZus{}pval}\PYG{p}{)}
\end{sphinxVerbatim}

\noindent{\hspace*{\fill}\sphinxincludegraphics[scale=0.6]{{volcano_example}.png}\hspace*{\fill}}

\end{description}

\end{itemize}

\end{fulllineitems}

\phantomsection\label{\detokenize{sets:module-data_tools.sets}}\index{data\_tools.sets (module)}

\section{data\_tools.sets}
\label{\detokenize{sets:data-tools-sets}}\label{\detokenize{sets::doc}}
Set operations module.


\subsection{Reference}
\label{\detokenize{sets:reference}}\index{bit\_or() (in module data\_tools.sets)}

\begin{fulllineitems}
\phantomsection\label{\detokenize{sets:data_tools.sets.bit_or}}\pysiglinewithargsret{\sphinxcode{data\_tools.sets.}\sphinxbfcode{bit\_or}}{\emph{a}, \emph{b}}{}
Returns the bit operation OR between two bit-strings \sphinxstyleemphasis{a} and \sphinxstyleemphasis{b}.
NOTE: \sphinxstyleemphasis{a} and \sphinxstyleemphasis{b} must have the same size.
\begin{itemize}
\item {} \begin{description}
\item[{Arguments:}] \leavevmode\begin{itemize}
\item {} 
\sphinxstyleemphasis{a} {[}tuple{]}: Or any iterable type.

\item {} 
\sphinxstyleemphasis{b} {[}tuple{]}: Or any iterable type.

\end{itemize}

\end{description}

\item {} \begin{description}
\item[{Returns:}] \leavevmode\begin{itemize}
\item {} 
{[}tuple{]}: OR operation between \sphinxstyleemphasis{a} and \sphinxstyleemphasis{b} element-wise.

\end{itemize}

\end{description}

\item {} \begin{description}
\item[{Examples:}] \leavevmode
\fvset{hllines={, ,}}%
\begin{sphinxVerbatim}[commandchars=\\\{\}]
\PYG{g+gp}{\PYGZgt{}\PYGZgt{}\PYGZgt{} }\PYG{n}{a}\PYG{p}{,} \PYG{n}{b} \PYG{o}{=} \PYG{p}{(}\PYG{l+m+mi}{0}\PYG{p}{,} \PYG{l+m+mi}{0}\PYG{p}{,} \PYG{l+m+mi}{1}\PYG{p}{)}\PYG{p}{,} \PYG{p}{(}\PYG{l+m+mi}{1}\PYG{p}{,} \PYG{l+m+mi}{0}\PYG{p}{,} \PYG{l+m+mi}{1}\PYG{p}{)}
\PYG{g+gp}{\PYGZgt{}\PYGZgt{}\PYGZgt{} }\PYG{n}{bit\PYGZus{}or}\PYG{p}{(}\PYG{n}{a}\PYG{p}{,} \PYG{n}{b}\PYG{p}{)}
\PYG{g+go}{(1, 0, 1)}
\end{sphinxVerbatim}

\end{description}

\end{itemize}

\end{fulllineitems}

\index{find\_min() (in module data\_tools.sets)}

\begin{fulllineitems}
\phantomsection\label{\detokenize{sets:data_tools.sets.find_min}}\pysiglinewithargsret{\sphinxcode{data\_tools.sets.}\sphinxbfcode{find\_min}}{\emph{A}}{}
Finds and returns the subset of vectors whose sum is minimum from a
given set \sphinxstyleemphasis{A}.
\begin{itemize}
\item {} \begin{description}
\item[{Arguments:}] \leavevmode\begin{itemize}
\item {} 
\sphinxstyleemphasis{A} {[}set{]}: Set of vectors ({[}tuple{]} or any iterable).

\end{itemize}

\end{description}

\item {} \begin{description}
\item[{Returns:}] \leavevmode\begin{itemize}
\item {} 
{[}set{]}: Subset of vectors in \sphinxstyleemphasis{A} whose sum is minimum.

\end{itemize}

\end{description}

\item {} \begin{description}
\item[{Examples:}] \leavevmode
\fvset{hllines={, ,}}%
\begin{sphinxVerbatim}[commandchars=\\\{\}]
\PYG{g+gp}{\PYGZgt{}\PYGZgt{}\PYGZgt{} }\PYG{n}{A} \PYG{o}{=} \PYG{p}{\PYGZob{}}\PYG{p}{(}\PYG{l+m+mi}{0}\PYG{p}{,} \PYG{l+m+mi}{1}\PYG{p}{,} \PYG{l+m+mi}{1}\PYG{p}{)}\PYG{p}{,} \PYG{p}{(}\PYG{l+m+mi}{0}\PYG{p}{,} \PYG{l+m+mi}{1}\PYG{p}{,} \PYG{l+m+mi}{0}\PYG{p}{)}\PYG{p}{,} \PYG{p}{(}\PYG{l+m+mi}{1}\PYG{p}{,} \PYG{l+m+mi}{0}\PYG{p}{,} \PYG{l+m+mi}{0}\PYG{p}{)}\PYG{p}{,} \PYG{p}{(}\PYG{l+m+mi}{1}\PYG{p}{,} \PYG{l+m+mi}{1}\PYG{p}{,} \PYG{l+m+mi}{1}\PYG{p}{)}\PYG{p}{\PYGZcb{}}
\PYG{g+gp}{\PYGZgt{}\PYGZgt{}\PYGZgt{} }\PYG{n}{find\PYGZus{}min}\PYG{p}{(}\PYG{n}{A}\PYG{p}{)}
\PYG{g+go}{set([(0, 1, 0), (1, 0, 0)])}
\end{sphinxVerbatim}

\end{description}

\end{itemize}

\end{fulllineitems}

\index{in\_all() (in module data\_tools.sets)}

\begin{fulllineitems}
\phantomsection\label{\detokenize{sets:data_tools.sets.in_all}}\pysiglinewithargsret{\sphinxcode{data\_tools.sets.}\sphinxbfcode{in\_all}}{\emph{x}, \emph{N}}{}
Checks if a vector \sphinxstyleemphasis{x} is present in all sets contained in a list
\sphinxstyleemphasis{N}.
\begin{itemize}
\item {} \begin{description}
\item[{Arguments:}] \leavevmode\begin{itemize}
\item {} 
\sphinxstyleemphasis{x} {[}tuple{]}: Or any hashable type as long as is the same
contained in the sets of \sphinxstyleemphasis{N}.

\item {} 
\sphinxstyleemphasis{N} {[}list{]}: Or any iterable type containing {[}set{]} objects.

\end{itemize}

\end{description}

\item {} \begin{description}
\item[{Returns:}] \leavevmode\begin{itemize}
\item {} 
{[}bool{]}: \sphinxcode{True} if \sphinxstyleemphasis{x} is found in all sets of \sphinxstyleemphasis{N}, \sphinxcode{False}
otherwise.

\end{itemize}

\end{description}

\item {} \begin{description}
\item[{Examples:}] \leavevmode
\fvset{hllines={, ,}}%
\begin{sphinxVerbatim}[commandchars=\\\{\}]
\PYG{g+gp}{\PYGZgt{}\PYGZgt{}\PYGZgt{} }\PYG{n}{N} \PYG{o}{=} \PYG{p}{[}\PYG{p}{\PYGZob{}}\PYG{p}{(}\PYG{l+m+mi}{0}\PYG{p}{,} \PYG{l+m+mi}{0}\PYG{p}{)}\PYG{p}{,} \PYG{p}{(}\PYG{l+m+mi}{0}\PYG{p}{,} \PYG{l+m+mi}{1}\PYG{p}{)}\PYG{p}{\PYGZcb{}}\PYG{p}{,} \PYG{c+c1}{\PYGZsh{} \PYGZlt{}\PYGZhy{} set A}
\PYG{g+gp}{... }     \PYG{p}{\PYGZob{}}\PYG{p}{(}\PYG{l+m+mi}{0}\PYG{p}{,} \PYG{l+m+mi}{0}\PYG{p}{)}\PYG{p}{,} \PYG{p}{(}\PYG{l+m+mi}{1}\PYG{p}{,} \PYG{l+m+mi}{1}\PYG{p}{)}\PYG{p}{,} \PYG{p}{(}\PYG{l+m+mi}{1}\PYG{p}{,} \PYG{l+m+mi}{0}\PYG{p}{)}\PYG{p}{\PYGZcb{}}\PYG{p}{]} \PYG{c+c1}{\PYGZsh{} \PYGZlt{}\PYGZhy{} set B}
\PYG{g+gp}{\PYGZgt{}\PYGZgt{}\PYGZgt{} }\PYG{n}{x} \PYG{o}{=} \PYG{p}{(}\PYG{l+m+mi}{0}\PYG{p}{,} \PYG{l+m+mi}{0}\PYG{p}{)}
\PYG{g+gp}{\PYGZgt{}\PYGZgt{}\PYGZgt{} }\PYG{n}{in\PYGZus{}all}\PYG{p}{(}\PYG{n}{x}\PYG{p}{,} \PYG{n}{N}\PYG{p}{)}
\PYG{g+go}{True}
\PYG{g+gp}{\PYGZgt{}\PYGZgt{}\PYGZgt{} }\PYG{n}{y} \PYG{o}{=} \PYG{p}{(}\PYG{l+m+mi}{0}\PYG{p}{,} \PYG{l+m+mi}{1}\PYG{p}{)}
\PYG{g+gp}{\PYGZgt{}\PYGZgt{}\PYGZgt{} }\PYG{n}{in\PYGZus{}all}\PYG{p}{(}\PYG{n}{y}\PYG{p}{,} \PYG{n}{N}\PYG{p}{)}
\PYG{g+go}{False}
\end{sphinxVerbatim}

\end{description}

\end{itemize}

\end{fulllineitems}

\index{subsets() (in module data\_tools.sets)}

\begin{fulllineitems}
\phantomsection\label{\detokenize{sets:data_tools.sets.subsets}}\pysiglinewithargsret{\sphinxcode{data\_tools.sets.}\sphinxbfcode{subsets}}{\emph{N}}{}
Function that computes all possible logical relations between all
sets on a list \sphinxstyleemphasis{N} and returns all subsets. This is, the subsets
that would represent each intersecting area on a Venn diagram.
\begin{itemize}
\item {} \begin{description}
\item[{Arguments:}] \leavevmode\begin{itemize}
\item {} 
\sphinxstyleemphasis{N} {[}list{]}: Or any iterable type containing {[}set{]} objects.

\end{itemize}

\end{description}

\item {} \begin{description}
\item[{Returns:}] \leavevmode\begin{itemize}
\item {} 
{[}dict{]}: Collection of subsets according to the logical
relations between the sets in \sphinxstyleemphasis{N}. The keys are binary codes
that denote the logical relation (see examples below).

\end{itemize}

\end{description}

\item {} \begin{description}
\item[{Examples:}] \leavevmode
\fvset{hllines={, ,}}%
\begin{sphinxVerbatim}[commandchars=\\\{\}]
\PYG{g+gp}{\PYGZgt{}\PYGZgt{}\PYGZgt{} }\PYG{n}{N} \PYG{o}{=} \PYG{p}{[}\PYG{p}{\PYGZob{}}\PYG{l+m+mi}{0}\PYG{p}{,} \PYG{l+m+mi}{1}\PYG{p}{,} \PYG{l+m+mi}{2}\PYG{p}{\PYGZcb{}}\PYG{p}{,} \PYG{p}{\PYGZob{}}\PYG{l+m+mi}{2}\PYG{p}{,} \PYG{l+m+mi}{3}\PYG{p}{,} \PYG{l+m+mi}{4}\PYG{p}{\PYGZcb{}}\PYG{p}{]}
\PYG{g+gp}{\PYGZgt{}\PYGZgt{}\PYGZgt{} }\PYG{n}{subsets}\PYG{p}{(}\PYG{n}{N}\PYG{p}{)}
\PYG{g+go}{\PYGZob{}\PYGZsq{}11\PYGZsq{}: set([2]), \PYGZsq{}10\PYGZsq{}: set([0, 1]), \PYGZsq{}01\PYGZsq{}: set([3, 4])\PYGZcb{}}
\PYG{g+gp}{\PYGZgt{}\PYGZgt{}\PYGZgt{} }\PYG{n}{N} \PYG{o}{=} \PYG{p}{[}\PYG{p}{\PYGZob{}}\PYG{l+m+mi}{0}\PYG{p}{,} \PYG{l+m+mi}{1}\PYG{p}{\PYGZcb{}}\PYG{p}{,} \PYG{p}{\PYGZob{}}\PYG{l+m+mi}{2}\PYG{p}{,} \PYG{l+m+mi}{3}\PYG{p}{\PYGZcb{}}\PYG{p}{,} \PYG{p}{\PYGZob{}}\PYG{l+m+mi}{1}\PYG{p}{,} \PYG{l+m+mi}{3}\PYG{p}{,} \PYG{l+m+mi}{4}\PYG{p}{\PYGZcb{}}\PYG{p}{]}
\PYG{g+gp}{\PYGZgt{}\PYGZgt{}\PYGZgt{} }\PYG{n}{subsets}\PYG{p}{(}\PYG{n}{N}\PYG{p}{)}
\PYG{g+go}{\PYGZob{}\PYGZsq{}010\PYGZsq{}: set([2]), \PYGZsq{}011\PYGZsq{}: set([3]), \PYGZsq{}001\PYGZsq{}: set([4]), \PYGZsq{}111\PYGZsq{}: set([}
\PYG{g+go}{]), \PYGZsq{}110\PYGZsq{}: set([]), \PYGZsq{}100\PYGZsq{}: set([0]), \PYGZsq{}101\PYGZsq{}: set([1])\PYGZcb{}}
\end{sphinxVerbatim}

\end{description}

\end{itemize}

\end{fulllineitems}

\phantomsection\label{\detokenize{strings:module-data_tools.strings}}\index{data\_tools.strings (module)}

\section{data\_tools.strings}
\label{\detokenize{strings::doc}}\label{\detokenize{strings:data-tools-strings}}
String operations module.


\subsection{Reference}
\label{\detokenize{strings:reference}}\index{is\_numeric() (in module data\_tools.strings)}

\begin{fulllineitems}
\phantomsection\label{\detokenize{strings:data_tools.strings.is_numeric}}\pysiglinewithargsret{\sphinxcode{data\_tools.strings.}\sphinxbfcode{is\_numeric}}{\emph{s}}{}
Determines if a string can be considered a numeric value. NaN is
also considered, since it is float type.
\begin{itemize}
\item {} \begin{description}
\item[{Arguments:}] \leavevmode\begin{itemize}
\item {} 
\sphinxstyleemphasis{s} {[}str{]}: String to be evaluated.

\end{itemize}

\end{description}

\item {} \begin{description}
\item[{Returns:}] \leavevmode\begin{itemize}
\item {} 
{[}bool{]}: \sphinxcode{True}/\sphinxcode{False} depending if the condition is
satisfied.

\end{itemize}

\end{description}

\item {} \begin{description}
\item[{Examples:}] \leavevmode
\fvset{hllines={, ,}}%
\begin{sphinxVerbatim}[commandchars=\\\{\}]
\PYG{g+gp}{\PYGZgt{}\PYGZgt{}\PYGZgt{} }\PYG{n}{is\PYGZus{}numeric}\PYG{p}{(}\PYG{l+s+s1}{\PYGZsq{}}\PYG{l+s+s1}{4}\PYG{l+s+s1}{\PYGZsq{}}\PYG{p}{)}
\PYG{g+go}{True}
\PYG{g+gp}{\PYGZgt{}\PYGZgt{}\PYGZgt{} }\PYG{n}{is\PYGZus{}numeric}\PYG{p}{(}\PYG{l+s+s1}{\PYGZsq{}}\PYG{l+s+s1}{\PYGZhy{}3.2}\PYG{l+s+s1}{\PYGZsq{}}\PYG{p}{)}
\PYG{g+go}{True}
\PYG{g+gp}{\PYGZgt{}\PYGZgt{}\PYGZgt{} }\PYG{n}{is\PYGZus{}numeric}\PYG{p}{(}\PYG{l+s+s1}{\PYGZsq{}}\PYG{l+s+s1}{number}\PYG{l+s+s1}{\PYGZsq{}}\PYG{p}{)}
\PYG{g+go}{False}
\PYG{g+gp}{\PYGZgt{}\PYGZgt{}\PYGZgt{} }\PYG{n}{is\PYGZus{}numeric}\PYG{p}{(}\PYG{l+s+s1}{\PYGZsq{}}\PYG{l+s+s1}{NaN}\PYG{l+s+s1}{\PYGZsq{}}\PYG{p}{)}
\PYG{g+go}{True}
\end{sphinxVerbatim}

\end{description}

\end{itemize}

\end{fulllineitems}

\index{join\_str\_lists() (in module data\_tools.strings)}

\begin{fulllineitems}
\phantomsection\label{\detokenize{strings:data_tools.strings.join_str_lists}}\pysiglinewithargsret{\sphinxcode{data\_tools.strings.}\sphinxbfcode{join\_str\_lists}}{\emph{a}, \emph{b}, \emph{sep=''}}{}
Joins element-wise two lists (or any 1D iterable) of strings with a
given separator (if provided). Length of the input lists must be
equal.
\begin{itemize}
\item {} \begin{description}
\item[{Arguments:}] \leavevmode\begin{itemize}
\item {} 
\sphinxstyleemphasis{a} {[}list{]}: Contains the first elements {[}str{]} of the joint
strings.

\item {} 
\sphinxstyleemphasis{b} {[}list{]}: Contains the second elements {[}str{]} of the joint
strings.

\item {} 
\sphinxstyleemphasis{sep} {[}str{]}: Optional \sphinxcode{'{'}} (non separated) by default.
Determines the separator between the joint strings.

\end{itemize}

\end{description}

\item {} \begin{description}
\item[{Returns:}] \leavevmode\begin{itemize}
\item {} 
{[}list{]}: List of the joint strings.

\end{itemize}

\end{description}

\item {} \begin{description}
\item[{Example:}] \leavevmode
\fvset{hllines={, ,}}%
\begin{sphinxVerbatim}[commandchars=\\\{\}]
\PYG{g+gp}{\PYGZgt{}\PYGZgt{}\PYGZgt{} }\PYG{n}{a} \PYG{o}{=} \PYG{p}{[}\PYG{l+s+s1}{\PYGZsq{}}\PYG{l+s+s1}{a}\PYG{l+s+s1}{\PYGZsq{}}\PYG{p}{,} \PYG{l+s+s1}{\PYGZsq{}}\PYG{l+s+s1}{b}\PYG{l+s+s1}{\PYGZsq{}}\PYG{p}{]}
\PYG{g+gp}{\PYGZgt{}\PYGZgt{}\PYGZgt{} }\PYG{n}{b} \PYG{o}{=} \PYG{p}{[}\PYG{l+s+s1}{\PYGZsq{}}\PYG{l+s+s1}{1}\PYG{l+s+s1}{\PYGZsq{}}\PYG{p}{,} \PYG{l+s+s1}{\PYGZsq{}}\PYG{l+s+s1}{2}\PYG{l+s+s1}{\PYGZsq{}}\PYG{p}{]}
\PYG{g+gp}{\PYGZgt{}\PYGZgt{}\PYGZgt{} }\PYG{n}{join\PYGZus{}str\PYGZus{}lists}\PYG{p}{(}\PYG{n}{a}\PYG{p}{,} \PYG{n}{b}\PYG{p}{,} \PYG{n}{sep}\PYG{o}{=}\PYG{l+s+s1}{\PYGZsq{}}\PYG{l+s+s1}{\PYGZus{}}\PYG{l+s+s1}{\PYGZsq{}}\PYG{p}{)}
\PYG{g+go}{[\PYGZsq{}a\PYGZus{}1\PYGZsq{}, \PYGZsq{}b\PYGZus{}2\PYGZsq{}]}
\end{sphinxVerbatim}

\end{description}

\end{itemize}

\end{fulllineitems}



\renewcommand{\indexname}{Python Module Index}
\begin{sphinxtheindex}
\def\bigletter#1{{\Large\sffamily#1}\nopagebreak\vspace{1mm}}
\bigletter{d}
\item {\sphinxstyleindexentry{data\_tools.diffusion}}\sphinxstyleindexpageref{diffusion:\detokenize{module-data_tools.diffusion}}
\item {\sphinxstyleindexentry{data\_tools.models}}\sphinxstyleindexpageref{models:\detokenize{module-data_tools.models}}
\item {\sphinxstyleindexentry{data\_tools.plots}}\sphinxstyleindexpageref{plots:\detokenize{module-data_tools.plots}}
\item {\sphinxstyleindexentry{data\_tools.sets}}\sphinxstyleindexpageref{sets:\detokenize{module-data_tools.sets}}
\item {\sphinxstyleindexentry{data\_tools.strings}}\sphinxstyleindexpageref{strings:\detokenize{module-data_tools.strings}}
\end{sphinxtheindex}

\renewcommand{\indexname}{Index}
\printindex
\end{document}